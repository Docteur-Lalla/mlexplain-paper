\documentclass[twocolumn]{article}
\usepackage[utf8]{inputenc}

\author{
	K\'evin Le Bon
	\and
	Alan Schmitt
}

\title{MLExplain}

\begin{document}
\maketitle

\begin{abstract}
	MLExplain is a step-by-step visual interpreter for OCaml that enables the user
	to inspect both their program's state and the interpreter's state itself.
\end{abstract}

\section{Introduction}

The semantics of a programming language can be very complex. When a language has no specification,
the semantics is then defined by the implementations of the language, i.e. interpreters and
compilers. However even when a specification is available, it can be difficult to understand why
the execution of a specific program results to a certain output.

The project \emph{JSExplain}\footnote{https://github.com/jscert/jsexplain} aims to describe
a JavaScript program's execution by showing every step of an interpreter which behavior is the
closest to JavaScript specification. This paper shows how we adapted JSExplain to OCaml.

Unlike JavaScript, OCaml has no official specification. However, OCaml code execution is
fairly simple because the amount of language constructions is small. We have written an
interpreter for OCaml's typed abstract syntax tree (AST). This AST is close to source code and
need little transformations but supplies us with resolved names -- which is useful for module and
signatures inclusion and absolutely mandatory for features like named parameters in functions.
The semantic we give to OCaml is higher level than the one described in \emph{ZINC}
\cite{Leroy-ZINC} virtual machine which is at the same level as object code.

\bibliographystyle{plain}
\bibliography{biblio}

\end{document}
